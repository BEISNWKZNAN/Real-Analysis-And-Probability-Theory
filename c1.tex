\chapter{抽象积分}

\section{回顾Riemann积分}
首先我们知道Riemann积分的定义
$$\int_{a}^{b} f(x) \,dx=\sum_{n = 1}^{n}f(t_{i})m(E_{i}) $$
其中 $m(E_{i})$总是一个区间的长度(测度),将其与函数的某个值相乘以得到函数所围面积的估计。即以函数的图以及$x=a,x=b,y=0$为边界点的图形(集合)的面积(测度)。更进一步的,我们有在学习重积分时定义的Jordan测度。
\begin{definition}{Jordan测度}
设D是$\mathbb{R}^{n}$中的有界集,则存在$I=\{x\in \mathbb{R}^{n}:a_{i}\leqslant x_{i}\leqslant b_{i},i=1,2,\cdots ,n\},I\supseteq D$,则定义函数$\mathcal{X}_D:I\to \mathbb{R},x\to \left\{
    \begin{array}{ll}
        1,x\in D \\
        0,x\notin D 
    \end{array}
\right.$.若函数$\mathcal{X}_{D}$是可积的,则称D为Jordan可测集,$\int_{I}\mathcal{X}_{D}(x)dx$定义为D的Jordan测度。
\end{definition}
这个测度的定义实际上是用内部的小矩形去逼近集合的面积。显然,I的选取并不影响D的可测性以及测度,所以这个定义是好的。
\par
同时我们可以看到集函数m一样是一个测度,不过他似乎只能定义在区间上。
\begin{definition}{区间测度}
若I是$\mathbb{R} $上的有界区间,即$I=(a,b),(a,b],[a,b)$或$[a,b],a\leqslant b,a,b\in \mathbb{R}$,则定义$m(I)=b-a$.
\par
若I是无界区间,则$m(I)=+\infty. $
\end{definition}
通过这两个已知的测度,我们应该可以认识到测度应该有的一些性质。并试图得到衡量一个集合大小的一般理论。
\section{测度的概念}
首先,测度应该是一个映射到$[0,\infty]$函数。
\par
其次,我们是试图衡量面积、长度之类的东西,所以他应该是要保留一些并集的性质,他应该可以需要保持有限并,即有有限可加性。进一步的,我们由集合论的知识可以知道对于势相同无穷集。可数并似乎不改变集合的势,比如,可数的直线不可能覆盖一个平面。或许我们可以加入可数可加性。即如果$A_{i}$是互不相交的集合列,应有
$$\mu (\bigcup_{n = 1}^{\infty}A_{i})=\sum_{n = 1}^{\infty} \mu(A_{i})$$
\par 
不过还有一个问题,刚刚我们并没有定义集函数定义域,我们称其定义域里的元素为可测集,为使定义域可以承受可数可加性,我们应该要求他的定义域是对于可数可加封闭的,我们通常认为实直线是无穷长的,所以全集X应该也在其定义域里,这样我们也可以要求任何可测集的补集也为可测的。这样我们就得到了$\sigma$-代数。
\begin{definition}
    我们称一个集合族$\mathfrak{M}$是$\sigma$代数如果他有如下性质。
    \par
    (i)   $X\in \mathfrak{M} $
    \par 
    (ii)  $A\in \mathfrak{M}\Rightarrow A^{c}\in \mathfrak{M}$
    \par 
    (iii) ($A_{i}\in \mathfrak{M})\vee(A = \bigcup_{n = 1}^{\infty}A_{i})\Rightarrow A\in \mathfrak{M} $
    \par
    如果一个集合属于$\mathfrak{M}$则称这个集合是可测的。
\end{definition}
这个定义与拓扑和开集的定义是相似的。开集的定义在数学分析中已经给出了。
\par 
接下我们或许可以给出一些测度的性质,以确认我们的定义的合适的。
\begin{proposition}
    如果$\mu$是定义在$\mathfrak{M}$上的一个测度,那么它有如下性质:
    \par 
    (1)
\end{proposition}